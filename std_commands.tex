%%%%%%%%%%%%%%%%%%%%%%%%%%%%%%%%%%%%%%%%%%%%%%%%%%%%%%%%%%%%%%%%%%%%%%

%-- DISCLAIMER:
%-- This is open sourced LaTeX utils that I adapted and changed over time,
%-- anyone are free to change or use it.

%%%%%%%%%%%%%%%%%%%%%%%%%%%%%%%%%%%%%%%%%%%%%%%%%%%%%%%%%%%%%%%%%%%%%%

%-- place any standard commands/environments here to get included in
%-- documents.  When you include this file, you should do it before
%-- the \begin{document} tag.

%-- NECESSARY ENVIRONMENTS OR PACKAGES (paste in to main.tex if necessay):


% \usepackage[usenames,dvipsnames,xcdraw,svgnames,table]{xcolor}
% \usepackage{array}
% \usepackage{ifthen}
% \usepackage{xspace}
% \usepackage{pifont}
% \usepackage{footnote}
% \usepackage{footmisc}
%
% % enumerate* and itemize*
% \usepackage[inline]{enumitem}
%
% % for \begin{comment}
% \usepackage{verbatim}
%
% % for source-code listings
% \usepackage[newfloat,draft=false]{minted}
%
% % for formulas
% \usepackage{mathtools, nccmath}
% \usepackage{xpatch}

% % to split lists into multiple columns
% \usepackage{multicol}
% https://tex.stackexchange.com/questions/166263/drawing-multicolumn-table-in-latex
% TODO: siunitx

%
% % for \sfrac
% \usepackage{xfrac}

% % Change from 1 to 0 to get rid of the comments.
% \newcommand{\showcomments}{1}
% %%%%%%%%%%%%%%%%%%%%%%%%%%%%%%%%%%%%%%%%%%%%%%%%%%%%%%%%%%%%%%%%%%%%%%

%-- DISCLAIMER:
%-- This is open sourced LaTeX utils that I adapted and changed over time,
%-- anyone are free to change or use it.

%%%%%%%%%%%%%%%%%%%%%%%%%%%%%%%%%%%%%%%%%%%%%%%%%%%%%%%%%%%%%%%%%%%%%%

%-- place any standard commands/environments here to get included in
%-- documents.  When you include this file, you should do it before
%-- the \begin{document} tag.

%-- NECESSARY ENVIRONMENTS OR PACKAGES (paste in to main.tex if necessay):


% \usepackage[usenames,dvipsnames,xcdraw,svgnames,table]{xcolor}
% \usepackage{array}
% \usepackage{ifthen}
% \usepackage{xspace}
% \usepackage{pifont}
% \usepackage{footnote}
% \usepackage{footmisc}
%
% % enumerate* and itemize*
% \usepackage[inline]{enumitem}
%
% % for \begin{comment}
% \usepackage{verbatim}
%
% % for source-code listings
% \usepackage[newfloat,draft=false]{minted}
%
% % for formulas
% \usepackage{mathtools, nccmath}
% \usepackage{xpatch}

% % to split lists into multiple columns
% \usepackage{multicol}
% https://tex.stackexchange.com/questions/166263/drawing-multicolumn-table-in-latex
% TODO: siunitx

%
% % for \sfrac
% \usepackage{xfrac}

% % Change from 1 to 0 to get rid of the comments.
% \newcommand{\showcomments}{1}
% %%%%%%%%%%%%%%%%%%%%%%%%%%%%%%%%%%%%%%%%%%%%%%%%%%%%%%%%%%%%%%%%%%%%%%

%-- DISCLAIMER:
%-- This is open sourced LaTeX utils that I adapted and changed over time,
%-- anyone are free to change or use it.

%%%%%%%%%%%%%%%%%%%%%%%%%%%%%%%%%%%%%%%%%%%%%%%%%%%%%%%%%%%%%%%%%%%%%%

%-- place any standard commands/environments here to get included in
%-- documents.  When you include this file, you should do it before
%-- the \begin{document} tag.

%-- NECESSARY ENVIRONMENTS OR PACKAGES (paste in to main.tex if necessay):


% \usepackage[usenames,dvipsnames,xcdraw,svgnames,table]{xcolor}
% \usepackage{array}
% \usepackage{ifthen}
% \usepackage{xspace}
% \usepackage{pifont}
% \usepackage{footnote}
% \usepackage{footmisc}
%
% % enumerate* and itemize*
% \usepackage[inline]{enumitem}
%
% % for \begin{comment}
% \usepackage{verbatim}
%
% % for source-code listings
% \usepackage[newfloat,draft=false]{minted}
%
% % for formulas
% \usepackage{mathtools, nccmath}
% \usepackage{xpatch}

% % to split lists into multiple columns
% \usepackage{multicol}
% https://tex.stackexchange.com/questions/166263/drawing-multicolumn-table-in-latex
% TODO: siunitx

%
% % for \sfrac
% \usepackage{xfrac}

% % Change from 1 to 0 to get rid of the comments.
% \newcommand{\showcomments}{1}
% %%%%%%%%%%%%%%%%%%%%%%%%%%%%%%%%%%%%%%%%%%%%%%%%%%%%%%%%%%%%%%%%%%%%%%

%-- DISCLAIMER:
%-- This is open sourced LaTeX utils that I adapted and changed over time,
%-- anyone are free to change or use it.

%%%%%%%%%%%%%%%%%%%%%%%%%%%%%%%%%%%%%%%%%%%%%%%%%%%%%%%%%%%%%%%%%%%%%%

%-- place any standard commands/environments here to get included in
%-- documents.  When you include this file, you should do it before
%-- the \begin{document} tag.

%-- NECESSARY ENVIRONMENTS OR PACKAGES (paste in to main.tex if necessay):


% \usepackage[usenames,dvipsnames,xcdraw,svgnames,table]{xcolor}
% \usepackage{array}
% \usepackage{ifthen}
% \usepackage{xspace}
% \usepackage{pifont}
% \usepackage{footnote}
% \usepackage{footmisc}
%
% % enumerate* and itemize*
% \usepackage[inline]{enumitem}
%
% % for \begin{comment}
% \usepackage{verbatim}
%
% % for source-code listings
% \usepackage[newfloat,draft=false]{minted}
%
% % for formulas
% \usepackage{mathtools, nccmath}
% \usepackage{xpatch}

% % to split lists into multiple columns
% \usepackage{multicol}
% https://tex.stackexchange.com/questions/166263/drawing-multicolumn-table-in-latex
% TODO: siunitx

%
% % for \sfrac
% \usepackage{xfrac}

% % Change from 1 to 0 to get rid of the comments.
% \newcommand{\showcomments}{1}
% \input{std_commands}
% % macro for project name
% \newcommand{\proj}{Rumpelstiltskin\xspace}
% \hyphenation{Rumpelstiltskin}

%%%%%%%%%%%%%%%%%%%%%%%%%%%%%%%%%%%%%%%%%%%%%%%%%%%%%%%%%%%%%%%%%%%%%%

% Defined values

% brewer qualitative colors
% \definecolor{brewergreen}{HTML}{1b9e77}
% \definecolor{brewerorange}{HTML}{d95f02}
% \definecolor{brewerpurple}{HTML}{7570b3}

% special foot note stuff
\newcommand{\astfootnote}[1]{
\let\oldthefootnote=\thefootnote
\setcounter{footnote}{0}
\renewcommand{\thefootnote}{\fnsymbol{footnote}}
\footnote{#1}
\let\thefootnote=\oldthefootnote
}

% a command to indicate current editing progress
\newcommand{\resume}{
  \begin{center}
    \color{brewerorange}
    \vspace{3pt}
    \hrule
    \vspace{1pt}
    \hrule
    \hrule
    \vspace{10pt}
    \textbf{This section is not yet complete.}
    \vspace{10pt}
    \hrule
    \hrule
    \vspace{1pt}
    \hrule
    \vspace{3pt}
  \end{center}
}

% reduce space before and after formula
% \xpatchcmd{\NCC@ignorepar}{%
% \abovedisplayskip\abovedisplayshortskip}
% {%
% \abovedisplayskip\abovedisplayshortskip%
% \belowdisplayskip\belowdisplayshortskip}
% {}{}
\setlength{\abovedisplayskip}{3pt} \setlength{\abovedisplayshortskip}{3pt}
\setlength{\belowdisplayskip}{5pt} \setlength{\belowdisplayshortskip}{5pt}

%%%%%%%%%%%%%%%%%%%%%%%%%%%%%%%%%%%%%%%%%%%%%%%%%%%%%%%%%%%%%%%%%%%%%%

% Common setup

% an environment for invariants
\newcounter{invn}
\renewcommand{\theinvn}{\Roman{invn}}
\newenvironment{invariant}{\vspace{.5em} \color{brewerorange} \refstepcounter{invn} \noindent \textbf{\color{brewerorange} Invariant \Roman{invn}.}}{\vspace{.5em}}

% for handy reference
%
% paragraph without spacing:
% \setparsizes{0pt}{0pt}{0pt plus 1fil}

% additional hyphenation rules
\hyphenation{da-ta-flow}

%-- Macros... examples...
\newcommand{\netex}{FlexNet\xspace}
\hyphenation{FlexNet}
\newcommand{\blah}{blah blah blah}

% an environment for todos
\newenvironment{inprogress}{\color{ACMBlue}\textbf{Sun:} \begin{tabular}{||p{.48\textwidth}}}{\end{tabular}}

\newcommand{\inprog}[1]{
  \ifthenelse{\equal{\showcomments}{1}}{{\color{ACMOrange}\vspace{.5em}%
  \noindent\begin{tabular}{@{}||p{.46\textwidth}} \textbf{In progress: } #1
\end{tabular}}}{}}
\newcommand{\replace}[1]{
  \ifthenelse{\equal{\showcomments}{1}}{{\color{VioletRed}\vspace{.5em}%\hspace*{.15em}
  \noindent\begin{tabular}{@{}||p{.46\textwidth}} \textbf{Replaced: } #1
\end{tabular}}}{}}
\newcommand{\update}[1]{
  \ifthenelse{\equal{\showcomments}{1}}{{\color{ACMBlue}\vspace{.5em}%
  \noindent\begin{tabular}{@{}||p{.46\textwidth}} \textbf{Update: } #1
\end{tabular}}}{}}


%-- Code... Not sure this is useful...
\newcommand{\code}[1]{\texttt{\textbf{#1}}}

%-- Terms...  Use this to introduce a term in the paper.
\newcommand{\term}[1]{\emph{#1}}

%-- Provides stylization for e-mail addresses
%\newcommand{\email}[1]{\emph{(#1)}}

%-- Starts a minor section (puts the title inline w/ the text.
\newcommand{\minorsection}[1]{\noindent\textbf{#1}:}
\newcommand{\minorsectionit}[1]{\noindent\textit{#1}:}

%-- Jiri caption
\newcommand{\minicaption}[2]{\caption[#1]{\textbf{#1} \textnormal{#2}}}

%-- Units on numbers: 4KB -> \units{4}{KB}
\newcommand{\units}[2]{#1\,#2}

%-- Commands...  i.e. WRITE commands.
\newcommand{\command}[1]{{\sc \MakeLowercase{#1}}}

%-- For notes about things that need to be fixed.
\newcommand{\fix}[1]{\marginpar{\LARGE\ensuremath{\bullet}}
\MakeUppercase{\textbf{[#1]}}}
%-- For adding inline notes to a draft preceded by your initials
%-- E.g., \fixnote{JJW}{What the heck is a foobar?}
\newcommand{\fixnote}[2]{\marginpar{\LARGE\ensuremath{\bullet}}{\textbf{[#1:} \textit{#2\,}\textbf{]}}}

%-- Setting margins: \setmargins{left}{right}{top}{bottom}
\newcommand{\setmargins}[4]{
  % Calculations of top & bottom margins
  \setlength\topmargin{#3}
  \addtolength\topmargin{-.5in}  %-- seems like this should be 1, but .5
  %-- balances the text top to bottom
  \addtolength\topmargin{-\headheight}
  \addtolength\topmargin{-\headsep}
  \setlength\textheight{\paperheight}
  \addtolength\textheight{-#3}
  \addtolength\textheight{-#4}

  % Calculations of left & right margins
  \setlength\oddsidemargin{#1}
  \addtolength\oddsidemargin{-1in}
  \setlength\evensidemargin{\oddsidemargin}
  \setlength\textwidth{\paperwidth}
  \addtolength\textwidth{-#1}
  \addtolength\textwidth{-#2}
}

%-- For the tabularx environment... Using L, C, R as the column type
%-- will left, center, or right justify the text.
% \newcolumntype{L}{X}
\newcolumntype{L}[1]{>{\raggedright\arraybackslash}p{#1}}
\newcolumntype{C}{>{\centering\arraybackslash}X}
\newcolumntype{R}{>{\raggedleft\arraybackslash}X}
\newcolumntype{P}[1]{>{\centering\arraybackslash}p{#1}}
\newcommand{\ra}[1]{\renewcommand{\arraystretch}{#1}}

%-- To comment out a swatch of text, use \omitit{blah blah blah} or \eat{ xxx }
\long\def\omitit#1{}
\newcommand\eat[1]{}
% \newcommand\bs{\bigskip}
\newcommand{\para}[1]{{\vspace{4pt} \bf \noindent #1 \hspace{10pt}}}
\newcommand{\bs}[1]{{\vspace{4pt} \bf \noindent #1\xspace}}

%-- Inline title; useful for sub-sub-sections in which you don't want a separate
%-- line for the title.
\newcommand{\inlinesection}[1]{\smallskip\noindent{\textbf{#1.}}}

%-- todo notes
\newenvironment{outlineenv}{\par\color{black}}{\par}
\newenvironment{pagelenenv}{\par\color{red}}{\par}

%\newcommand{\outline}[1]{\begin{outlineenv}#1\end{outlineenv}}
\newcommand{\paralenblah}[1]{\begin{pagelenenv}Estimated length: #1 paragraphs\end{pagelenenv}}
\newcommand{\pagelenblah}[1]{\begin{pagelenenv}Estimated length: #1 pages\end{pagelenenv}}

\newcommand{\pagelen}[1]{\ifthenelse{\equal{\showcomments}{1}}{\pagelenblah{#1}}{}}
\newcommand{\outlinetext}[1]{\ifthenelse{\equal{\showcomments}{1}}{\outline{#1}}{}}

% define shortcuts for comment
% \newcommand{\outline}[1]{\textsf{\textbf{\leavevmode\color{VioletRed}[Outline: #1]}}}
\newcommand{\todoA}[1]{\textsf{\textbf{\color{BurntOrange}{[[#1]]}}}}
\newcommand{\todoB}[1]{\textsf{\textbf{\color{RoyalPurple}{[[#1]]}}}}
\newcommand{\todoC}[1]{\textsf{\textbf{\color{ForestGreen}{[[#1]]}}}}
\newcommand{\todoD}[1]{\textsf{\textbf{\color{VioletRed}{[[#1]]}}}}
\newcommand{\todoE}[1]{\textsf{\textbf{\color{RedOrange}{[[#1]]}}}}
\newcommand{\todoF}[1]{\textsf{\textbf{\color{RoyalBlue}{[[#1]]}}}}
\newcommand{\rough}[1]{\textit{\color{Orange}{#1}}}

% \newcommand{\pjd}[1]{\ifthenelse{\equal{\showcomments}{1}}{\textbf{\textcolor{RoyalBlue}{[[[Peter: #1]]]}}}
% \newcommand{\khor}[1]{\ifthenelse{\equal{\showcomments}{1}}{\textbf{\textcolor{BurntOrange}{[[[Isaac: #1]]]}}}
% \newcommand{\sun}[1]{\ifthenelse{\equal{\showcomments}{1}}{\textbf{\textcolor{ForestGreen}{[[[Shuwen: #1]]]}}}
\newcommand{\todo}[1]{\ifthenelse{\equal{\showcomments}{1}}{\todoA{TODO: #1}}{}}
\newcommand{\ma}[1]{\ifthenelse{\equal{\showcomments}{1}}{\todoC{MA: #1}}{}}
\newcommand{\edit}[1]{\ifthenelse{\equal{\showcomments}{1}}{\todoD{Edited: #1}}{}}

% eg, ie, etc, et al ....
% \newcommand{\cf}{{cf.}\xspace}
% \newcommand\etc{etc\@ifnextchar.{}{.\@}}
% \newcommand{\eg}{\emph{e.g.},\xspace}
\newcommand{\etal}{et al.\xspace}
% \newcommand{\ie}{\emph{i.e.}\xspace}
\newcommand{\re}{r.e.\xspace}
\newcommand{\aka}{a.k.a.\xspace}

\newcommand{\ie}{\emph{i.\,e}\@ifnextchar.{}{.\@\xspace}}
% exempli gratia
\newcommand{\eg}{\emph{e.\,g}\@ifnextchar.{}{.\@\xspace}}
% et cetera
\newcommand{\etc}{\emph{etc}\@ifnextchar.{}{.\@\xspace}}
% confer
\newcommand{\cf}{\emph{cf}\@ifnextchar.{}{.\@\xspace}}
% versus
\newcommand{\vs}{\emph{vs}\@ifnextchar.{}{.\@\xspace}}



% packed items and enum..
\newenvironment{packed_itemize}{
\begin{itemize}[leftmargin=*]
  \setlength{\itemsep}{2pt}
  \setlength{\parskip}{0pt}
  \setlength{\parsep}{0pt}
  % \setlength{\partopsep}{0pt}
  \setlength{\topsep}{2pt}
}{\end{itemize}}

\newenvironment{packed_enumerate}{
\begin{enumerate}[leftmargin=*]
  \setlength{\itemsep}{2pt}
  \setlength{\parskip}{0pt}
  \setlength{\parsep}{0pt}
  \setlength{\topsep}{2pt}
}{\end{enumerate}}

\def\compactify{\itemsep=0pt \topsep=0pt \partopsep=0pt \parsep=0pt}
  \let\latexusecounter=\usecounter
  \newenvironment{CompactItemize}
    {\def\usecounter{\compactify\latexusecounter}
     \begin{itemize}}
    {\end{itemize}\let\usecounter=\latexusecounter}
  \newenvironment{CompactEnumerate}
    {\def\usecounter{\compactify\latexusecounter}
     \begin{enumerate}}
    {\end{enumerate}\let\usecounter=\latexusecounter}

%%%%%%%%%%%%%%%%%%%%%%%%%%%%%%%%%%%%%%%%%%%%%%%%%%%%%%%%%%%%%%%%

% Little numbers in circles plus background colors
% \newcommand{\colorcirc}[2]{\protect\raisebox{-0.5pt}{\tikzexternaldisable\protect\tikz{\protect\path[fill=#1] (0,0.004) circle [radius=0.105];\protect\node [inner sep=0,outer sep=0] at (0,0) {\small{#2}};}}}
\newcommand{\colorcirc}[2]{\protect\raisebox{-0.5pt}{\protect\tikz{\protect\path[fill=#1] (0,0.004) circle [radius=0.105];\protect\node [inner sep=0,outer sep=0] at (0,0) {\small{#2}};}}}

%-- Little numbers in circles...  Maybe possible to define smarter macro, but
%-- simple attempts didn't work.  There are also UTF-8 characters that can be used.
%-- Might need this to get it working:
\newcommand{\circone}{\protect\raisebox{-0.5pt}{\ding{192}}\xspace}
\newcommand{\circtwo}{\protect\raisebox{-0.5pt}{\ding{193}}\xspace}
\newcommand{\circthree}{\protect\raisebox{-0.5pt}{\ding{194}}\xspace}
\newcommand{\circfour}{\protect\raisebox{-0.5pt}{\ding{195}}\xspace}
\newcommand{\circfive}{\protect\raisebox{-0.5pt}{\ding{196}}\xspace}
\newcommand{\circsix}{\protect\raisebox{-0.5pt}{\ding{197}}\xspace}
\newcommand{\circseven}{\protect\raisebox{-0.5pt}{\ding{198}}\xspace}
\newcommand{\circeight}{\protect\raisebox{-0.5pt}{\ding{199}}\xspace}
\newcommand{\circnine}{\protect\raisebox{-0.5pt}{\ding{200}}\xspace}
\newcommand{\circten}{\protect\raisebox{-0.5pt}{\ding{201}}\xspace}

%\renewcommand{\footnotelayout}{\setstretch{1.05}} % Reduce line spacing in footnote
\setlength{\skip\footins}{1mm} % Reduce space between main text and footnote
\setlength{\footnotesep}{1mm} % Reduce space between footnotes

%% This part goes in preamble
\newcommand{\dummyfig}[1]{
  \centering
  \fbox{
    \begin{minipage}[c][.45\textheight][c]{.95\textwidth}
      \centering{#1}
    \end{minipage}
  }
}

% \renewcommand{\footnotesize}{\scriptsize}
\newcommand{\squeezeup}{\vspace{-6.0mm}}

% tikz configuration
\usetikzlibrary{shapes.geometric,shapes.arrows,decorations.pathmorphing}
\usetikzlibrary{matrix,chains,scopes,positioning,arrows,fit}
\usetikzlibrary{arrows.meta, chains, positioning, shapes}
% \renewcommand{\rmdefault}{bch} % change default font
\tikzset{
  % STYLES
  node distance =0.5cm, auto,%font=\footnotesize,
  every node/.style={node distance=1cm},
  % The comment style is used to describe the characteristics of each force
  comment/.style={rectangle, inner sep=2pt, text width=2.5cm, node distance=0.25cm,}, %font=\scriptsize\sffamily},
  % The force style is used to draw the forces' name
  force/.style={rectangle, draw, fill=black!10, inner sep=2pt, text width=2.5cm, text badly centered, minimum height=1cm,}, %font=\bfseries\footnotesize\sffamily},
  %Define standard arrow tip
  >=stealth',
  %Define style for boxes
  punkt/.style={
    rectangle,
    rounded corners,
    draw=black, very thick,
    text width=3.5em,
    minimum height=2em,
  text centered},
  % Define arrow style
  pil/.style={
    ->,
    thick,
    shorten <=2pt,
  shorten >=2pt,},
  % colored ones
  gray/.style = {fill=black!10},
  blue/.style = {fill=blue!30},
  red/.style = {fill=red!30},
  green/.style = {fill=green!30},
  orange/.style = {fill=orange!15},
  % process/.style = {base, minimum width=2.5cm, fill=orange!15,font=\ttfamily},
}

% % macro for project name
% \newcommand{\proj}{Rumpelstiltskin\xspace}
% \hyphenation{Rumpelstiltskin}

%%%%%%%%%%%%%%%%%%%%%%%%%%%%%%%%%%%%%%%%%%%%%%%%%%%%%%%%%%%%%%%%%%%%%%

% Defined values

% brewer qualitative colors
% \definecolor{brewergreen}{HTML}{1b9e77}
% \definecolor{brewerorange}{HTML}{d95f02}
% \definecolor{brewerpurple}{HTML}{7570b3}

% special foot note stuff
\newcommand{\astfootnote}[1]{
\let\oldthefootnote=\thefootnote
\setcounter{footnote}{0}
\renewcommand{\thefootnote}{\fnsymbol{footnote}}
\footnote{#1}
\let\thefootnote=\oldthefootnote
}

% a command to indicate current editing progress
\newcommand{\resume}{
  \begin{center}
    \color{brewerorange}
    \vspace{3pt}
    \hrule
    \vspace{1pt}
    \hrule
    \hrule
    \vspace{10pt}
    \textbf{This section is not yet complete.}
    \vspace{10pt}
    \hrule
    \hrule
    \vspace{1pt}
    \hrule
    \vspace{3pt}
  \end{center}
}

% reduce space before and after formula
% \xpatchcmd{\NCC@ignorepar}{%
% \abovedisplayskip\abovedisplayshortskip}
% {%
% \abovedisplayskip\abovedisplayshortskip%
% \belowdisplayskip\belowdisplayshortskip}
% {}{}
\setlength{\abovedisplayskip}{3pt} \setlength{\abovedisplayshortskip}{3pt}
\setlength{\belowdisplayskip}{5pt} \setlength{\belowdisplayshortskip}{5pt}

%%%%%%%%%%%%%%%%%%%%%%%%%%%%%%%%%%%%%%%%%%%%%%%%%%%%%%%%%%%%%%%%%%%%%%

% Common setup

% an environment for invariants
\newcounter{invn}
\renewcommand{\theinvn}{\Roman{invn}}
\newenvironment{invariant}{\vspace{.5em} \color{brewerorange} \refstepcounter{invn} \noindent \textbf{\color{brewerorange} Invariant \Roman{invn}.}}{\vspace{.5em}}

% for handy reference
%
% paragraph without spacing:
% \setparsizes{0pt}{0pt}{0pt plus 1fil}

% additional hyphenation rules
\hyphenation{da-ta-flow}

%-- Macros... examples...
\newcommand{\netex}{FlexNet\xspace}
\hyphenation{FlexNet}
\newcommand{\blah}{blah blah blah}

% an environment for todos
\newenvironment{inprogress}{\color{ACMBlue}\textbf{Sun:} \begin{tabular}{||p{.48\textwidth}}}{\end{tabular}}

\newcommand{\inprog}[1]{
  \ifthenelse{\equal{\showcomments}{1}}{{\color{ACMOrange}\vspace{.5em}%
  \noindent\begin{tabular}{@{}||p{.46\textwidth}} \textbf{In progress: } #1
\end{tabular}}}{}}
\newcommand{\replace}[1]{
  \ifthenelse{\equal{\showcomments}{1}}{{\color{VioletRed}\vspace{.5em}%\hspace*{.15em}
  \noindent\begin{tabular}{@{}||p{.46\textwidth}} \textbf{Replaced: } #1
\end{tabular}}}{}}
\newcommand{\update}[1]{
  \ifthenelse{\equal{\showcomments}{1}}{{\color{ACMBlue}\vspace{.5em}%
  \noindent\begin{tabular}{@{}||p{.46\textwidth}} \textbf{Update: } #1
\end{tabular}}}{}}


%-- Code... Not sure this is useful...
\newcommand{\code}[1]{\texttt{\textbf{#1}}}

%-- Terms...  Use this to introduce a term in the paper.
\newcommand{\term}[1]{\emph{#1}}

%-- Provides stylization for e-mail addresses
%\newcommand{\email}[1]{\emph{(#1)}}

%-- Starts a minor section (puts the title inline w/ the text.
\newcommand{\minorsection}[1]{\noindent\textbf{#1}:}
\newcommand{\minorsectionit}[1]{\noindent\textit{#1}:}

%-- Jiri caption
\newcommand{\minicaption}[2]{\caption[#1]{\textbf{#1} \textnormal{#2}}}

%-- Units on numbers: 4KB -> \units{4}{KB}
\newcommand{\units}[2]{#1\,#2}

%-- Commands...  i.e. WRITE commands.
\newcommand{\command}[1]{{\sc \MakeLowercase{#1}}}

%-- For notes about things that need to be fixed.
\newcommand{\fix}[1]{\marginpar{\LARGE\ensuremath{\bullet}}
\MakeUppercase{\textbf{[#1]}}}
%-- For adding inline notes to a draft preceded by your initials
%-- E.g., \fixnote{JJW}{What the heck is a foobar?}
\newcommand{\fixnote}[2]{\marginpar{\LARGE\ensuremath{\bullet}}{\textbf{[#1:} \textit{#2\,}\textbf{]}}}

%-- Setting margins: \setmargins{left}{right}{top}{bottom}
\newcommand{\setmargins}[4]{
  % Calculations of top & bottom margins
  \setlength\topmargin{#3}
  \addtolength\topmargin{-.5in}  %-- seems like this should be 1, but .5
  %-- balances the text top to bottom
  \addtolength\topmargin{-\headheight}
  \addtolength\topmargin{-\headsep}
  \setlength\textheight{\paperheight}
  \addtolength\textheight{-#3}
  \addtolength\textheight{-#4}

  % Calculations of left & right margins
  \setlength\oddsidemargin{#1}
  \addtolength\oddsidemargin{-1in}
  \setlength\evensidemargin{\oddsidemargin}
  \setlength\textwidth{\paperwidth}
  \addtolength\textwidth{-#1}
  \addtolength\textwidth{-#2}
}

%-- For the tabularx environment... Using L, C, R as the column type
%-- will left, center, or right justify the text.
% \newcolumntype{L}{X}
\newcolumntype{L}[1]{>{\raggedright\arraybackslash}p{#1}}
\newcolumntype{C}{>{\centering\arraybackslash}X}
\newcolumntype{R}{>{\raggedleft\arraybackslash}X}
\newcolumntype{P}[1]{>{\centering\arraybackslash}p{#1}}
\newcommand{\ra}[1]{\renewcommand{\arraystretch}{#1}}

%-- To comment out a swatch of text, use \omitit{blah blah blah} or \eat{ xxx }
\long\def\omitit#1{}
\newcommand\eat[1]{}
% \newcommand\bs{\bigskip}
\newcommand{\para}[1]{{\vspace{4pt} \bf \noindent #1 \hspace{10pt}}}
\newcommand{\bs}[1]{{\vspace{4pt} \bf \noindent #1\xspace}}

%-- Inline title; useful for sub-sub-sections in which you don't want a separate
%-- line for the title.
\newcommand{\inlinesection}[1]{\smallskip\noindent{\textbf{#1.}}}

%-- todo notes
\newenvironment{outlineenv}{\par\color{black}}{\par}
\newenvironment{pagelenenv}{\par\color{red}}{\par}

%\newcommand{\outline}[1]{\begin{outlineenv}#1\end{outlineenv}}
\newcommand{\paralenblah}[1]{\begin{pagelenenv}Estimated length: #1 paragraphs\end{pagelenenv}}
\newcommand{\pagelenblah}[1]{\begin{pagelenenv}Estimated length: #1 pages\end{pagelenenv}}

\newcommand{\pagelen}[1]{\ifthenelse{\equal{\showcomments}{1}}{\pagelenblah{#1}}{}}
\newcommand{\outlinetext}[1]{\ifthenelse{\equal{\showcomments}{1}}{\outline{#1}}{}}

% define shortcuts for comment
% \newcommand{\outline}[1]{\textsf{\textbf{\leavevmode\color{VioletRed}[Outline: #1]}}}
\newcommand{\todoA}[1]{\textsf{\textbf{\color{BurntOrange}{[[#1]]}}}}
\newcommand{\todoB}[1]{\textsf{\textbf{\color{RoyalPurple}{[[#1]]}}}}
\newcommand{\todoC}[1]{\textsf{\textbf{\color{ForestGreen}{[[#1]]}}}}
\newcommand{\todoD}[1]{\textsf{\textbf{\color{VioletRed}{[[#1]]}}}}
\newcommand{\todoE}[1]{\textsf{\textbf{\color{RedOrange}{[[#1]]}}}}
\newcommand{\todoF}[1]{\textsf{\textbf{\color{RoyalBlue}{[[#1]]}}}}
\newcommand{\rough}[1]{\textit{\color{Orange}{#1}}}

% \newcommand{\pjd}[1]{\ifthenelse{\equal{\showcomments}{1}}{\textbf{\textcolor{RoyalBlue}{[[[Peter: #1]]]}}}
% \newcommand{\khor}[1]{\ifthenelse{\equal{\showcomments}{1}}{\textbf{\textcolor{BurntOrange}{[[[Isaac: #1]]]}}}
% \newcommand{\sun}[1]{\ifthenelse{\equal{\showcomments}{1}}{\textbf{\textcolor{ForestGreen}{[[[Shuwen: #1]]]}}}
\newcommand{\todo}[1]{\ifthenelse{\equal{\showcomments}{1}}{\todoA{TODO: #1}}{}}
\newcommand{\ma}[1]{\ifthenelse{\equal{\showcomments}{1}}{\todoC{MA: #1}}{}}
\newcommand{\edit}[1]{\ifthenelse{\equal{\showcomments}{1}}{\todoD{Edited: #1}}{}}

% eg, ie, etc, et al ....
% \newcommand{\cf}{{cf.}\xspace}
% \newcommand\etc{etc\@ifnextchar.{}{.\@}}
% \newcommand{\eg}{\emph{e.g.},\xspace}
\newcommand{\etal}{et al.\xspace}
% \newcommand{\ie}{\emph{i.e.}\xspace}
\newcommand{\re}{r.e.\xspace}
\newcommand{\aka}{a.k.a.\xspace}

\newcommand{\ie}{\emph{i.\,e}\@ifnextchar.{}{.\@\xspace}}
% exempli gratia
\newcommand{\eg}{\emph{e.\,g}\@ifnextchar.{}{.\@\xspace}}
% et cetera
\newcommand{\etc}{\emph{etc}\@ifnextchar.{}{.\@\xspace}}
% confer
\newcommand{\cf}{\emph{cf}\@ifnextchar.{}{.\@\xspace}}
% versus
\newcommand{\vs}{\emph{vs}\@ifnextchar.{}{.\@\xspace}}



% packed items and enum..
\newenvironment{packed_itemize}{
\begin{itemize}[leftmargin=*]
  \setlength{\itemsep}{2pt}
  \setlength{\parskip}{0pt}
  \setlength{\parsep}{0pt}
  % \setlength{\partopsep}{0pt}
  \setlength{\topsep}{2pt}
}{\end{itemize}}

\newenvironment{packed_enumerate}{
\begin{enumerate}[leftmargin=*]
  \setlength{\itemsep}{2pt}
  \setlength{\parskip}{0pt}
  \setlength{\parsep}{0pt}
  \setlength{\topsep}{2pt}
}{\end{enumerate}}

\def\compactify{\itemsep=0pt \topsep=0pt \partopsep=0pt \parsep=0pt}
  \let\latexusecounter=\usecounter
  \newenvironment{CompactItemize}
    {\def\usecounter{\compactify\latexusecounter}
     \begin{itemize}}
    {\end{itemize}\let\usecounter=\latexusecounter}
  \newenvironment{CompactEnumerate}
    {\def\usecounter{\compactify\latexusecounter}
     \begin{enumerate}}
    {\end{enumerate}\let\usecounter=\latexusecounter}

%%%%%%%%%%%%%%%%%%%%%%%%%%%%%%%%%%%%%%%%%%%%%%%%%%%%%%%%%%%%%%%%

% Little numbers in circles plus background colors
% \newcommand{\colorcirc}[2]{\protect\raisebox{-0.5pt}{\tikzexternaldisable\protect\tikz{\protect\path[fill=#1] (0,0.004) circle [radius=0.105];\protect\node [inner sep=0,outer sep=0] at (0,0) {\small{#2}};}}}
\newcommand{\colorcirc}[2]{\protect\raisebox{-0.5pt}{\protect\tikz{\protect\path[fill=#1] (0,0.004) circle [radius=0.105];\protect\node [inner sep=0,outer sep=0] at (0,0) {\small{#2}};}}}

%-- Little numbers in circles...  Maybe possible to define smarter macro, but
%-- simple attempts didn't work.  There are also UTF-8 characters that can be used.
%-- Might need this to get it working:
\newcommand{\circone}{\protect\raisebox{-0.5pt}{\ding{192}}\xspace}
\newcommand{\circtwo}{\protect\raisebox{-0.5pt}{\ding{193}}\xspace}
\newcommand{\circthree}{\protect\raisebox{-0.5pt}{\ding{194}}\xspace}
\newcommand{\circfour}{\protect\raisebox{-0.5pt}{\ding{195}}\xspace}
\newcommand{\circfive}{\protect\raisebox{-0.5pt}{\ding{196}}\xspace}
\newcommand{\circsix}{\protect\raisebox{-0.5pt}{\ding{197}}\xspace}
\newcommand{\circseven}{\protect\raisebox{-0.5pt}{\ding{198}}\xspace}
\newcommand{\circeight}{\protect\raisebox{-0.5pt}{\ding{199}}\xspace}
\newcommand{\circnine}{\protect\raisebox{-0.5pt}{\ding{200}}\xspace}
\newcommand{\circten}{\protect\raisebox{-0.5pt}{\ding{201}}\xspace}

%\renewcommand{\footnotelayout}{\setstretch{1.05}} % Reduce line spacing in footnote
\setlength{\skip\footins}{1mm} % Reduce space between main text and footnote
\setlength{\footnotesep}{1mm} % Reduce space between footnotes

%% This part goes in preamble
\newcommand{\dummyfig}[1]{
  \centering
  \fbox{
    \begin{minipage}[c][.45\textheight][c]{.95\textwidth}
      \centering{#1}
    \end{minipage}
  }
}

% \renewcommand{\footnotesize}{\scriptsize}
\newcommand{\squeezeup}{\vspace{-6.0mm}}

% tikz configuration
\usetikzlibrary{shapes.geometric,shapes.arrows,decorations.pathmorphing}
\usetikzlibrary{matrix,chains,scopes,positioning,arrows,fit}
\usetikzlibrary{arrows.meta, chains, positioning, shapes}
% \renewcommand{\rmdefault}{bch} % change default font
\tikzset{
  % STYLES
  node distance =0.5cm, auto,%font=\footnotesize,
  every node/.style={node distance=1cm},
  % The comment style is used to describe the characteristics of each force
  comment/.style={rectangle, inner sep=2pt, text width=2.5cm, node distance=0.25cm,}, %font=\scriptsize\sffamily},
  % The force style is used to draw the forces' name
  force/.style={rectangle, draw, fill=black!10, inner sep=2pt, text width=2.5cm, text badly centered, minimum height=1cm,}, %font=\bfseries\footnotesize\sffamily},
  %Define standard arrow tip
  >=stealth',
  %Define style for boxes
  punkt/.style={
    rectangle,
    rounded corners,
    draw=black, very thick,
    text width=3.5em,
    minimum height=2em,
  text centered},
  % Define arrow style
  pil/.style={
    ->,
    thick,
    shorten <=2pt,
  shorten >=2pt,},
  % colored ones
  gray/.style = {fill=black!10},
  blue/.style = {fill=blue!30},
  red/.style = {fill=red!30},
  green/.style = {fill=green!30},
  orange/.style = {fill=orange!15},
  % process/.style = {base, minimum width=2.5cm, fill=orange!15,font=\ttfamily},
}

% % macro for project name
% \newcommand{\proj}{Rumpelstiltskin\xspace}
% \hyphenation{Rumpelstiltskin}

%%%%%%%%%%%%%%%%%%%%%%%%%%%%%%%%%%%%%%%%%%%%%%%%%%%%%%%%%%%%%%%%%%%%%%

% Defined values

% brewer qualitative colors
% \definecolor{brewergreen}{HTML}{1b9e77}
% \definecolor{brewerorange}{HTML}{d95f02}
% \definecolor{brewerpurple}{HTML}{7570b3}

% special foot note stuff
\newcommand{\astfootnote}[1]{
\let\oldthefootnote=\thefootnote
\setcounter{footnote}{0}
\renewcommand{\thefootnote}{\fnsymbol{footnote}}
\footnote{#1}
\let\thefootnote=\oldthefootnote
}

% a command to indicate current editing progress
\newcommand{\resume}{
  \begin{center}
    \color{brewerorange}
    \vspace{3pt}
    \hrule
    \vspace{1pt}
    \hrule
    \hrule
    \vspace{10pt}
    \textbf{This section is not yet complete.}
    \vspace{10pt}
    \hrule
    \hrule
    \vspace{1pt}
    \hrule
    \vspace{3pt}
  \end{center}
}

% reduce space before and after formula
% \xpatchcmd{\NCC@ignorepar}{%
% \abovedisplayskip\abovedisplayshortskip}
% {%
% \abovedisplayskip\abovedisplayshortskip%
% \belowdisplayskip\belowdisplayshortskip}
% {}{}
\setlength{\abovedisplayskip}{3pt} \setlength{\abovedisplayshortskip}{3pt}
\setlength{\belowdisplayskip}{5pt} \setlength{\belowdisplayshortskip}{5pt}

%%%%%%%%%%%%%%%%%%%%%%%%%%%%%%%%%%%%%%%%%%%%%%%%%%%%%%%%%%%%%%%%%%%%%%

% Common setup

% an environment for invariants
\newcounter{invn}
\renewcommand{\theinvn}{\Roman{invn}}
\newenvironment{invariant}{\vspace{.5em} \color{brewerorange} \refstepcounter{invn} \noindent \textbf{\color{brewerorange} Invariant \Roman{invn}.}}{\vspace{.5em}}

% for handy reference
%
% paragraph without spacing:
% \setparsizes{0pt}{0pt}{0pt plus 1fil}

% additional hyphenation rules
\hyphenation{da-ta-flow}

%-- Macros... examples...
\newcommand{\netex}{FlexNet\xspace}
\hyphenation{FlexNet}
\newcommand{\blah}{blah blah blah}

% an environment for todos
\newenvironment{inprogress}{\color{ACMBlue}\textbf{Sun:} \begin{tabular}{||p{.48\textwidth}}}{\end{tabular}}

\newcommand{\inprog}[1]{
  \ifthenelse{\equal{\showcomments}{1}}{{\color{ACMOrange}\vspace{.5em}%
  \noindent\begin{tabular}{@{}||p{.46\textwidth}} \textbf{In progress: } #1
\end{tabular}}}{}}
\newcommand{\replace}[1]{
  \ifthenelse{\equal{\showcomments}{1}}{{\color{VioletRed}\vspace{.5em}%\hspace*{.15em}
  \noindent\begin{tabular}{@{}||p{.46\textwidth}} \textbf{Replaced: } #1
\end{tabular}}}{}}
\newcommand{\update}[1]{
  \ifthenelse{\equal{\showcomments}{1}}{{\color{ACMBlue}\vspace{.5em}%
  \noindent\begin{tabular}{@{}||p{.46\textwidth}} \textbf{Update: } #1
\end{tabular}}}{}}


%-- Code... Not sure this is useful...
\newcommand{\code}[1]{\texttt{\textbf{#1}}}

%-- Terms...  Use this to introduce a term in the paper.
\newcommand{\term}[1]{\emph{#1}}

%-- Provides stylization for e-mail addresses
%\newcommand{\email}[1]{\emph{(#1)}}

%-- Starts a minor section (puts the title inline w/ the text.
\newcommand{\minorsection}[1]{\noindent\textbf{#1}:}
\newcommand{\minorsectionit}[1]{\noindent\textit{#1}:}

%-- Jiri caption
\newcommand{\minicaption}[2]{\caption[#1]{\textbf{#1} \textnormal{#2}}}

%-- Units on numbers: 4KB -> \units{4}{KB}
\newcommand{\units}[2]{#1\,#2}

%-- Commands...  i.e. WRITE commands.
\newcommand{\command}[1]{{\sc \MakeLowercase{#1}}}

%-- For notes about things that need to be fixed.
\newcommand{\fix}[1]{\marginpar{\LARGE\ensuremath{\bullet}}
\MakeUppercase{\textbf{[#1]}}}
%-- For adding inline notes to a draft preceded by your initials
%-- E.g., \fixnote{JJW}{What the heck is a foobar?}
\newcommand{\fixnote}[2]{\marginpar{\LARGE\ensuremath{\bullet}}{\textbf{[#1:} \textit{#2\,}\textbf{]}}}

%-- Setting margins: \setmargins{left}{right}{top}{bottom}
\newcommand{\setmargins}[4]{
  % Calculations of top & bottom margins
  \setlength\topmargin{#3}
  \addtolength\topmargin{-.5in}  %-- seems like this should be 1, but .5
  %-- balances the text top to bottom
  \addtolength\topmargin{-\headheight}
  \addtolength\topmargin{-\headsep}
  \setlength\textheight{\paperheight}
  \addtolength\textheight{-#3}
  \addtolength\textheight{-#4}

  % Calculations of left & right margins
  \setlength\oddsidemargin{#1}
  \addtolength\oddsidemargin{-1in}
  \setlength\evensidemargin{\oddsidemargin}
  \setlength\textwidth{\paperwidth}
  \addtolength\textwidth{-#1}
  \addtolength\textwidth{-#2}
}

%-- For the tabularx environment... Using L, C, R as the column type
%-- will left, center, or right justify the text.
% \newcolumntype{L}{X}
\newcolumntype{L}[1]{>{\raggedright\arraybackslash}p{#1}}
\newcolumntype{C}{>{\centering\arraybackslash}X}
\newcolumntype{R}{>{\raggedleft\arraybackslash}X}
\newcolumntype{P}[1]{>{\centering\arraybackslash}p{#1}}
\newcommand{\ra}[1]{\renewcommand{\arraystretch}{#1}}

%-- To comment out a swatch of text, use \omitit{blah blah blah} or \eat{ xxx }
\long\def\omitit#1{}
\newcommand\eat[1]{}
% \newcommand\bs{\bigskip}
\newcommand{\para}[1]{{\vspace{4pt} \bf \noindent #1 \hspace{10pt}}}
\newcommand{\bs}[1]{{\vspace{4pt} \bf \noindent #1\xspace}}

%-- Inline title; useful for sub-sub-sections in which you don't want a separate
%-- line for the title.
\newcommand{\inlinesection}[1]{\smallskip\noindent{\textbf{#1.}}}

%-- todo notes
\newenvironment{outlineenv}{\par\color{black}}{\par}
\newenvironment{pagelenenv}{\par\color{red}}{\par}

%\newcommand{\outline}[1]{\begin{outlineenv}#1\end{outlineenv}}
\newcommand{\paralenblah}[1]{\begin{pagelenenv}Estimated length: #1 paragraphs\end{pagelenenv}}
\newcommand{\pagelenblah}[1]{\begin{pagelenenv}Estimated length: #1 pages\end{pagelenenv}}

\newcommand{\pagelen}[1]{\ifthenelse{\equal{\showcomments}{1}}{\pagelenblah{#1}}{}}
\newcommand{\outlinetext}[1]{\ifthenelse{\equal{\showcomments}{1}}{\outline{#1}}{}}

% define shortcuts for comment
% \newcommand{\outline}[1]{\textsf{\textbf{\leavevmode\color{VioletRed}[Outline: #1]}}}
\newcommand{\todoA}[1]{\textsf{\textbf{\color{BurntOrange}{[[#1]]}}}}
\newcommand{\todoB}[1]{\textsf{\textbf{\color{RoyalPurple}{[[#1]]}}}}
\newcommand{\todoC}[1]{\textsf{\textbf{\color{ForestGreen}{[[#1]]}}}}
\newcommand{\todoD}[1]{\textsf{\textbf{\color{VioletRed}{[[#1]]}}}}
\newcommand{\todoE}[1]{\textsf{\textbf{\color{RedOrange}{[[#1]]}}}}
\newcommand{\todoF}[1]{\textsf{\textbf{\color{RoyalBlue}{[[#1]]}}}}
\newcommand{\rough}[1]{\textit{\color{Orange}{#1}}}

% \newcommand{\pjd}[1]{\ifthenelse{\equal{\showcomments}{1}}{\textbf{\textcolor{RoyalBlue}{[[[Peter: #1]]]}}}
% \newcommand{\khor}[1]{\ifthenelse{\equal{\showcomments}{1}}{\textbf{\textcolor{BurntOrange}{[[[Isaac: #1]]]}}}
% \newcommand{\sun}[1]{\ifthenelse{\equal{\showcomments}{1}}{\textbf{\textcolor{ForestGreen}{[[[Shuwen: #1]]]}}}
\newcommand{\todo}[1]{\ifthenelse{\equal{\showcomments}{1}}{\todoA{TODO: #1}}{}}
\newcommand{\ma}[1]{\ifthenelse{\equal{\showcomments}{1}}{\todoC{MA: #1}}{}}
\newcommand{\edit}[1]{\ifthenelse{\equal{\showcomments}{1}}{\todoD{Edited: #1}}{}}

% eg, ie, etc, et al ....
% \newcommand{\cf}{{cf.}\xspace}
% \newcommand\etc{etc\@ifnextchar.{}{.\@}}
% \newcommand{\eg}{\emph{e.g.},\xspace}
\newcommand{\etal}{et al.\xspace}
% \newcommand{\ie}{\emph{i.e.}\xspace}
\newcommand{\re}{r.e.\xspace}
\newcommand{\aka}{a.k.a.\xspace}

\newcommand{\ie}{\emph{i.\,e}\@ifnextchar.{}{.\@\xspace}}
% exempli gratia
\newcommand{\eg}{\emph{e.\,g}\@ifnextchar.{}{.\@\xspace}}
% et cetera
\newcommand{\etc}{\emph{etc}\@ifnextchar.{}{.\@\xspace}}
% confer
\newcommand{\cf}{\emph{cf}\@ifnextchar.{}{.\@\xspace}}
% versus
\newcommand{\vs}{\emph{vs}\@ifnextchar.{}{.\@\xspace}}



% packed items and enum..
\newenvironment{packed_itemize}{
\begin{itemize}[leftmargin=*]
  \setlength{\itemsep}{2pt}
  \setlength{\parskip}{0pt}
  \setlength{\parsep}{0pt}
  % \setlength{\partopsep}{0pt}
  \setlength{\topsep}{2pt}
}{\end{itemize}}

\newenvironment{packed_enumerate}{
\begin{enumerate}[leftmargin=*]
  \setlength{\itemsep}{2pt}
  \setlength{\parskip}{0pt}
  \setlength{\parsep}{0pt}
  \setlength{\topsep}{2pt}
}{\end{enumerate}}

\def\compactify{\itemsep=0pt \topsep=0pt \partopsep=0pt \parsep=0pt}
  \let\latexusecounter=\usecounter
  \newenvironment{CompactItemize}
    {\def\usecounter{\compactify\latexusecounter}
     \begin{itemize}}
    {\end{itemize}\let\usecounter=\latexusecounter}
  \newenvironment{CompactEnumerate}
    {\def\usecounter{\compactify\latexusecounter}
     \begin{enumerate}}
    {\end{enumerate}\let\usecounter=\latexusecounter}

%%%%%%%%%%%%%%%%%%%%%%%%%%%%%%%%%%%%%%%%%%%%%%%%%%%%%%%%%%%%%%%%

% Little numbers in circles plus background colors
% \newcommand{\colorcirc}[2]{\protect\raisebox{-0.5pt}{\tikzexternaldisable\protect\tikz{\protect\path[fill=#1] (0,0.004) circle [radius=0.105];\protect\node [inner sep=0,outer sep=0] at (0,0) {\small{#2}};}}}
\newcommand{\colorcirc}[2]{\protect\raisebox{-0.5pt}{\protect\tikz{\protect\path[fill=#1] (0,0.004) circle [radius=0.105];\protect\node [inner sep=0,outer sep=0] at (0,0) {\small{#2}};}}}

%-- Little numbers in circles...  Maybe possible to define smarter macro, but
%-- simple attempts didn't work.  There are also UTF-8 characters that can be used.
%-- Might need this to get it working:
\newcommand{\circone}{\protect\raisebox{-0.5pt}{\ding{192}}\xspace}
\newcommand{\circtwo}{\protect\raisebox{-0.5pt}{\ding{193}}\xspace}
\newcommand{\circthree}{\protect\raisebox{-0.5pt}{\ding{194}}\xspace}
\newcommand{\circfour}{\protect\raisebox{-0.5pt}{\ding{195}}\xspace}
\newcommand{\circfive}{\protect\raisebox{-0.5pt}{\ding{196}}\xspace}
\newcommand{\circsix}{\protect\raisebox{-0.5pt}{\ding{197}}\xspace}
\newcommand{\circseven}{\protect\raisebox{-0.5pt}{\ding{198}}\xspace}
\newcommand{\circeight}{\protect\raisebox{-0.5pt}{\ding{199}}\xspace}
\newcommand{\circnine}{\protect\raisebox{-0.5pt}{\ding{200}}\xspace}
\newcommand{\circten}{\protect\raisebox{-0.5pt}{\ding{201}}\xspace}

%\renewcommand{\footnotelayout}{\setstretch{1.05}} % Reduce line spacing in footnote
\setlength{\skip\footins}{1mm} % Reduce space between main text and footnote
\setlength{\footnotesep}{1mm} % Reduce space between footnotes

%% This part goes in preamble
\newcommand{\dummyfig}[1]{
  \centering
  \fbox{
    \begin{minipage}[c][.45\textheight][c]{.95\textwidth}
      \centering{#1}
    \end{minipage}
  }
}

% \renewcommand{\footnotesize}{\scriptsize}
\newcommand{\squeezeup}{\vspace{-6.0mm}}

% tikz configuration
\usetikzlibrary{shapes.geometric,shapes.arrows,decorations.pathmorphing}
\usetikzlibrary{matrix,chains,scopes,positioning,arrows,fit}
\usetikzlibrary{arrows.meta, chains, positioning, shapes}
% \renewcommand{\rmdefault}{bch} % change default font
\tikzset{
  % STYLES
  node distance =0.5cm, auto,%font=\footnotesize,
  every node/.style={node distance=1cm},
  % The comment style is used to describe the characteristics of each force
  comment/.style={rectangle, inner sep=2pt, text width=2.5cm, node distance=0.25cm,}, %font=\scriptsize\sffamily},
  % The force style is used to draw the forces' name
  force/.style={rectangle, draw, fill=black!10, inner sep=2pt, text width=2.5cm, text badly centered, minimum height=1cm,}, %font=\bfseries\footnotesize\sffamily},
  %Define standard arrow tip
  >=stealth',
  %Define style for boxes
  punkt/.style={
    rectangle,
    rounded corners,
    draw=black, very thick,
    text width=3.5em,
    minimum height=2em,
  text centered},
  % Define arrow style
  pil/.style={
    ->,
    thick,
    shorten <=2pt,
  shorten >=2pt,},
  % colored ones
  gray/.style = {fill=black!10},
  blue/.style = {fill=blue!30},
  red/.style = {fill=red!30},
  green/.style = {fill=green!30},
  orange/.style = {fill=orange!15},
  % process/.style = {base, minimum width=2.5cm, fill=orange!15,font=\ttfamily},
}

% % macro for project name
% \newcommand{\sysname}{ZStore\xspace}
% \hyphenation{Rumpelstiltskin}

%%%%%%%%%%%%%%%%%%%%%%%%%%%%%%%%%%%%%%%%%%%%%%%%%%%%%%%%%%%%%%%%%%%%%%

% Defined values

% brewer qualitative colors
% \definecolor{brewergreen}{HTML}{1b9e77}
% \definecolor{brewerorange}{HTML}{d95f02}
% \definecolor{brewerpurple}{HTML}{7570b3}

% special foot note stuff
\newcommand{\astfootnote}[1]{
\let\oldthefootnote=\thefootnote
\setcounter{footnote}{0}
\renewcommand{\thefootnote}{\fnsymbol{footnote}}
\footnote{#1}
\let\thefootnote=\oldthefootnote
}

% a command to indicate current editing progress
\newcommand{\resume}{
  \begin{center}
    \color{brewerorange}
    \vspace{3pt}
    \hrule
    \vspace{1pt}
    \hrule
    \hrule
    \vspace{10pt}
    \textbf{This section is not yet complete.}
    \vspace{10pt}
    \hrule
    \hrule
    \vspace{1pt}
    \hrule
    \vspace{3pt}
  \end{center}
}

% reduce space before and after formula
% \xpatchcmd{\NCC@ignorepar}{%
% \abovedisplayskip\abovedisplayshortskip}
% {%
% \abovedisplayskip\abovedisplayshortskip%
% \belowdisplayskip\belowdisplayshortskip}
% {}{}
\setlength{\abovedisplayskip}{3pt} \setlength{\abovedisplayshortskip}{3pt}
\setlength{\belowdisplayskip}{5pt} \setlength{\belowdisplayshortskip}{5pt}

%%%%%%%%%%%%%%%%%%%%%%%%%%%%%%%%%%%%%%%%%%%%%%%%%%%%%%%%%%%%%%%%%%%%%%

% Common setup

% an environment for invariants
\newcounter{invn}
\renewcommand{\theinvn}{\Roman{invn}}
\newenvironment{invariant}{\vspace{.5em} \color{brewerorange} \refstepcounter{invn} \noindent \textbf{\color{brewerorange} Invariant \Roman{invn}.}}{\vspace{.5em}}

% for handy reference
%
% paragraph without spacing:
% \setparsizes{0pt}{0pt}{0pt plus 1fil}

% additional hyphenation rules
\hyphenation{da-ta-flow}

%-- Macros... examples...
% \newcommand{\netex}{FlexNet\xspace}
% \hyphenation{FlexNet}
\newcommand{\blah}{blah blah blah}

% an environment for todos
\newenvironment{inprogress}{\color{ACMBlue}\textbf{Sun:} \begin{tabular}{||p{.48\textwidth}}}{\end{tabular}}

\newcommand{\inprog}[1]{
  \ifthenelse{\equal{\showcomments}{1}}{{\color{ACMOrange}\vspace{.5em}%
  \noindent\begin{tabular}{@{}||p{.46\textwidth}} \textbf{In progress: } #1
\end{tabular}}}{}}
\newcommand{\replace}[1]{
  \ifthenelse{\equal{\showcomments}{1}}{{\color{VioletRed}\vspace{.5em}%\hspace*{.15em}
  \noindent\begin{tabular}{@{}||p{.46\textwidth}} \textbf{Replaced: } #1
\end{tabular}}}{}}
\newcommand{\update}[1]{
  \ifthenelse{\equal{\showcomments}{1}}{{\color{ACMBlue}\vspace{.5em}%
  \noindent\begin{tabular}{@{}||p{.46\textwidth}} \textbf{Update: } #1
\end{tabular}}}{}}


%-- Code... Not sure this is useful...
\newcommand{\code}[1]{\texttt{\textbf{#1}}}

%-- Terms...  Use this to introduce a term in the paper.
\newcommand{\term}[1]{\emph{#1}}

%-- Provides stylization for e-mail addresses
%\newcommand{\email}[1]{\emph{(#1)}}

%-- Starts a minor section (puts the title inline w/ the text.
\newcommand{\minorsection}[1]{\noindent\textbf{#1}:}
\newcommand{\minorsectionit}[1]{\noindent\textit{#1}:}

%-- Jiri caption
\newcommand{\minicaption}[2]{\caption[#1]{\textbf{#1} \textnormal{#2}}}

%-- Units on numbers: 4KB -> \units{4}{KB}
\newcommand{\units}[2]{#1\,#2}

%-- Commands...  i.e. WRITE commands.
\newcommand{\command}[1]{{\sc \MakeLowercase{#1}}}

%-- For notes about things that need to be fixed.
\newcommand{\fix}[1]{\marginpar{\LARGE\ensuremath{\bullet}}
\MakeUppercase{\textbf{[#1]}}}
%-- For adding inline notes to a draft preceded by your initials
%-- E.g., \fixnote{JJW}{What the heck is a foobar?}
\newcommand{\fixnote}[2]{\marginpar{\LARGE\ensuremath{\bullet}}{\textbf{[#1:} \textit{#2\,}\textbf{]}}}

%-- Setting margins: \setmargins{left}{right}{top}{bottom}
\newcommand{\setmargins}[4]{
  % Calculations of top & bottom margins
  \setlength\topmargin{#3}
  \addtolength\topmargin{-.5in}  %-- seems like this should be 1, but .5
  %-- balances the text top to bottom
  \addtolength\topmargin{-\headheight}
  \addtolength\topmargin{-\headsep}
  \setlength\textheight{\paperheight}
  \addtolength\textheight{-#3}
  \addtolength\textheight{-#4}

  % Calculations of left & right margins
  \setlength\oddsidemargin{#1}
  \addtolength\oddsidemargin{-1in}
  \setlength\evensidemargin{\oddsidemargin}
  \setlength\textwidth{\paperwidth}
  \addtolength\textwidth{-#1}
  \addtolength\textwidth{-#2}
}

%-- For the tabularx environment... Using L, C, R as the column type
%-- will left, center, or right justify the text.
% \newcolumntype{L}{X}
\newcolumntype{L}[1]{>{\raggedright\arraybackslash}p{#1}}
\newcolumntype{C}{>{\centering\arraybackslash}X}
\newcolumntype{R}{>{\raggedleft\arraybackslash}X}
\newcolumntype{P}[1]{>{\centering\arraybackslash}p{#1}}
\newcommand{\ra}[1]{\renewcommand{\arraystretch}{#1}}

%-- To comment out a swatch of text, use \omitit{blah blah blah} or \eat{ xxx }
\long\def\omitit#1{}
\newcommand\eat[1]{}
% \newcommand\bs{\bigskip}
\newcommand{\para}[1]{{\vspace{4pt} \bf \noindent #1 \hspace{10pt}}}
\newcommand{\bs}[1]{{\vspace{4pt} \bf \noindent #1\xspace}}

%-- Inline title; useful for sub-sub-sections in which you don't want a separate
%-- line for the title.
\newcommand{\inlinesection}[1]{\smallskip\noindent{\textbf{#1.}}}

%-- todo notes
\newenvironment{outlineenv}{\par\color{black}}{\par}
\newenvironment{pagelenenv}{\par\color{red}}{\par}

%\newcommand{\outline}[1]{\begin{outlineenv}#1\end{outlineenv}}
\newcommand{\paralenblah}[1]{\begin{pagelenenv}Estimated length: #1 paragraphs\end{pagelenenv}}
\newcommand{\pagelenblah}[1]{\begin{pagelenenv}Estimated length: #1 pages\end{pagelenenv}}

\newcommand{\pagelen}[1]{\ifthenelse{\equal{\showcomments}{1}}{\pagelenblah{#1}}{}}
\newcommand{\outlinetext}[1]{\ifthenelse{\equal{\showcomments}{1}}{\outline{#1}}{}}

% define shortcuts for comment
% \newcommand{\outline}[1]{\textsf{\textbf{\leavevmode\color{VioletRed}[Outline: #1]}}}
\newcommand{\todoA}[1]{\textsf{\textbf{\color{BurntOrange}{[[#1]]}}}}
\newcommand{\todoB}[1]{\textsf{\textbf{\color{RoyalPurple}{[[#1]]}}}}
\newcommand{\todoC}[1]{\textsf{\textbf{\color{ForestGreen}{[[#1]]}}}}
\newcommand{\todoD}[1]{\textsf{\textbf{\color{VioletRed}{[[#1]]}}}}
\newcommand{\todoE}[1]{\textsf{\textbf{\color{RedOrange}{[[#1]]}}}}
\newcommand{\todoF}[1]{\textsf{\textbf{\color{RoyalBlue}{[[#1]]}}}}
\newcommand{\rough}[1]{\textit{\color{Orange}{#1}}}

% \newcommand{\pjd}[1]{\ifthenelse{\equal{\showcomments}{1}}{\textbf{\textcolor{RoyalBlue}{[[[Peter: #1]]]}}}
% \newcommand{\khor}[1]{\ifthenelse{\equal{\showcomments}{1}}{\textbf{\textcolor{BurntOrange}{[[[Isaac: #1]]]}}}
% \newcommand{\sun}[1]{\ifthenelse{\equal{\showcomments}{1}}{\textbf{\textcolor{ForestGreen}{[[[Shuwen: #1]]]}}}
\newcommand{\todo}[1]{\ifthenelse{\equal{\showcomments}{1}}{\todoA{TODO: #1}}{}}
\newcommand{\ma}[1]{\ifthenelse{\equal{\showcomments}{1}}{\todoC{MA: #1}}{}}
\newcommand{\edit}[1]{\ifthenelse{\equal{\showcomments}{1}}{\todoD{Edited: #1}}{}}

% eg, ie, etc, et al ....
\newcommand{\cf}{{cf.}\xspace}
\newcommand\etc{etc\@ifnextchar.{}{.\@}}
\newcommand{\eg}{\emph{e.g.},\xspace}
\newcommand{\etal}{et al.\xspace}
\newcommand{\ie}{\emph{i.e.}\xspace}
\newcommand{\re}{r.e.\xspace}
\newcommand{\aka}{a.k.a.\xspace}

% packed items and enum..
\newenvironment{packed_itemize}{
\begin{itemize}[leftmargin=*]
  \setlength{\itemsep}{2pt}
  \setlength{\parskip}{0pt}
  \setlength{\parsep}{0pt}
  % \setlength{\partopsep}{0pt}
  \setlength{\topsep}{2pt}
}{\end{itemize}}

\newenvironment{packed_enumerate}{
\begin{enumerate}[leftmargin=*]
  \setlength{\itemsep}{2pt}
  \setlength{\parskip}{0pt}
  \setlength{\parsep}{0pt}
  \setlength{\topsep}{2pt}
}{\end{enumerate}}

\def\compactify{\itemsep=0pt \topsep=0pt \partopsep=0pt \parsep=0pt}
  \let\latexusecounter=\usecounter
  \newenvironment{CompactItemize}
    {\def\usecounter{\compactify\latexusecounter}
     \begin{itemize}}
    {\end{itemize}\let\usecounter=\latexusecounter}
  \newenvironment{CompactEnumerate}
    {\def\usecounter{\compactify\latexusecounter}
     \begin{enumerate}}
    {\end{enumerate}\let\usecounter=\latexusecounter}

%%%%%%%%%%%%%%%%%%%%%%%%%%%%%%%%%%%%%%%%%%%%%%%%%%%%%%%%%%%%%%%%

% Little numbers in circles plus background colors
% \newcommand{\colorcirc}[2]{\protect\raisebox{-0.5pt}{\tikzexternaldisable\protect\tikz{\protect\path[fill=#1] (0,0.004) circle [radius=0.105];\protect\node [inner sep=0,outer sep=0] at (0,0) {\small{#2}};}}}
\newcommand{\colorcirc}[2]{\protect\raisebox{-0.5pt}{\protect\tikz{\protect\path[fill=#1] (0,0.004) circle [radius=0.105];\protect\node [inner sep=0,outer sep=0] at (0,0) {\small{#2}};}}}

%-- Little numbers in circles...  Maybe possible to define smarter macro, but
%-- simple attempts didn't work.  There are also UTF-8 characters that can be used.
%-- Might need this to get it working:
\newcommand{\circone}{\protect\raisebox{-0.5pt}{\ding{192}}\xspace}
\newcommand{\circtwo}{\protect\raisebox{-0.5pt}{\ding{193}}\xspace}
\newcommand{\circthree}{\protect\raisebox{-0.5pt}{\ding{194}}\xspace}
\newcommand{\circfour}{\protect\raisebox{-0.5pt}{\ding{195}}\xspace}
\newcommand{\circfive}{\protect\raisebox{-0.5pt}{\ding{196}}\xspace}
\newcommand{\circsix}{\protect\raisebox{-0.5pt}{\ding{197}}\xspace}
\newcommand{\circseven}{\protect\raisebox{-0.5pt}{\ding{198}}\xspace}
\newcommand{\circeight}{\protect\raisebox{-0.5pt}{\ding{199}}\xspace}
\newcommand{\circnine}{\protect\raisebox{-0.5pt}{\ding{200}}\xspace}
\newcommand{\circten}{\protect\raisebox{-0.5pt}{\ding{201}}\xspace}

\newcommand{\postfigspace}{\vspace{-1.5em}}

\newcommand{\fakepara}[1]{\vspace{2mm}\noindent\textbf{#1}\hspace{5mm}}
\newcommand{\eg}{\textit{e.g.}\xspace}
\newcommand{\ie}{\textit{i.e.}\xspace}
\newcommand{\cf}{cf.\xspace}
\newcommand{\apriori}{\textit{a priori}\xspace}
\newcommand{\algname}{\textsc{Mainspring}} %


%\renewcommand{\footnotelayout}{\setstretch{1.05}} % Reduce line spacing in footnote
\setlength{\skip\footins}{1mm} % Reduce space between main text and footnote
\setlength{\footnotesep}{1mm} % Reduce space between footnotes

%% This part goes in preamble
\newcommand{\dummyfig}[1]{
  \centering
  \fbox{
    \begin{minipage}[c][.45\textheight][c]{.95\textwidth}
      \centering{#1}
    \end{minipage}
  }
}

% \renewcommand{\footnotesize}{\scriptsize}
\newcommand{\squeezeup}{\vspace{-6.0mm}}

% tikz configuration
\usetikzlibrary{shapes.geometric,shapes.arrows,decorations.pathmorphing}
\usetikzlibrary{matrix,chains,scopes,positioning,arrows,fit}
\usetikzlibrary{arrows.meta, chains, positioning, shapes}
% \renewcommand{\rmdefault}{bch} % change default font
\tikzset{
  % STYLES
  node distance =0.5cm, auto,%font=\footnotesize,
  every node/.style={node distance=1cm},
  % The comment style is used to describe the characteristics of each force
  comment/.style={rectangle, inner sep=2pt, text width=2.5cm, node distance=0.25cm,}, %font=\scriptsize\sffamily},
  % The force style is used to draw the forces' name
  force/.style={rectangle, draw, fill=black!10, inner sep=2pt, text width=2.5cm, text badly centered, minimum height=1cm,}, %font=\bfseries\footnotesize\sffamily},
  %Define standard arrow tip
  >=stealth',
  %Define style for boxes
  punkt/.style={
    rectangle,
    rounded corners,
    draw=black, very thick,
    text width=3.5em,
    minimum height=2em,
  text centered},
  % Define arrow style
  pil/.style={
    ->,
    thick,
    shorten <=2pt,
  shorten >=2pt,},
  % colored ones
  gray/.style = {fill=black!10},
  blue/.style = {fill=blue!30},
  red/.style = {fill=red!30},
  green/.style = {fill=green!30},
  orange/.style = {fill=orange!15},
  % process/.style = {base, minimum width=2.5cm, fill=orange!15,font=\ttfamily},
}
